With \textit{Tree Shaking}, static bundlers such as Webpack and RollupJS provide a service that helps to automatically remove unused code segments from the compiled JavaScript files. As part of this thesis, difficulties and best practices related to \textit{Tree Shaking} are identified. This is done with the assistance of a literature search. Further insights are gained by analysing JavaScript projects on GitHub that successfully implement \textit{Tree Shaking}.

Static code analysis, or linting, helps developers to identify bugs in the program code before actually compiling the application. Subsequently, a plugin for the \textit{JavaScript Linter} ESLint will be created. It checks the requirements for successful \textit{Tree Shaking} and, if necessary, it displays warnings or errors for the user. A static code analysis with the \textit{Tree Shaking Plugin} is performed on six NPM libraries and the results are documented.

At the end of this work, conclusions will be drawn from the findings and an outlook on possible further research will be given.