\appendix

\textbf{\color{red} Anhänge löschen, die nicht verwendet werden.}

\section*{Appendix}
\addcontentsline{toc}{section}{Appendices}
\renewcommand{\thesubsection}{\Alph{subsection}}

\subsection{git-Repository}

Das Repository dient zur Dokumentation und Nachvollziehbarkeit der Arbeitsschritte. Stellen Sie sicher, dass der/die BetreuerIn Zugriff auf das Repository hat. Stellen im Sinne des Datenschutzes sicher, dass das Repository nicht für andere zugänglich ist.

Daten für Bachelorarbeit 1 und 2:

\begin{itemize}
	\item LaTeX-Code der finalen Version der Arbeit
	\item alle Publikationen, die als pdf verfügbar sind.
	\item alle Webseiten als pdf
\end{itemize}

Daten für Bachelorarbeit 2:
\begin{itemize}
	\item Quellcode für praktischen Teil
	\item Vorlagen für Studienmaterial (Fragebögen, Einverständniserklärung, ...)	
	\item eingescanntes, ausgefülltes Studienmaterial (Fragebögen, Einverständniserklärung, ...)
	\item Rohdaten und aufbereitete Daten der Evaluierungen (Log-Daten, Tabellen, Graphen, Scripts, ...)	
\end{itemize}

Link zum Repository auf dem MMT-git-Server {\url{gitlab.mediacube.at}}:

{\color{red}\url{https://gitlab.mediacube.at/fhs123456/Abschlussarbeiten-Max-Muster}}
	
\subsection{Vorlagen für Studienmaterial}

Vorlagen für Studienmaterial müssen in den Anhang. 

\subsection{Archivierte Webseiten}
% \show\UrlBreaks

\url{http://web.archive.org/web/20160526143921/http://www.gamedev.net/page/resources/_/technical/game-programming/understanding-component-entity-systems-r3013}, letzter Zugriff 1.1.2016

\url{http://web.archive.org/web/20160526144551/http://scottbilas.com/files/2002/gdc_san_jose/game_objects_slides_with_notes.pdf}, letzter Zugriff 1.1.2016


