Mit \textit{Tree Shaking} bieten \textit{Static Bundler} wie Webpack und RollupJS einen Service, welcher hilft nicht benötigte Code Segmente automatisch aus den kompilierten JavaScript Dateien zu entfernen. Als Teil dieser Arbeit werden Schwierigkeiten und \textit{Best Practices} im Zusammenhang mit \textit{Tree Shaking} ermittelt. Dies geschieht mittels einer Literaturrecherche. Weitere Erkenntnisse werden durch die Untersuchung von JavaScript Projekten auf GitHub, welche \textit{Tree Shaking} erfolgreich verwenden, gewonnen.

Die statische Code Analyse, oder auch \textit{Linting}, hilft Entwicklern und Entwicklerinnen Fehler im Programmcode vor dem Kompilieren der Anwendung zu erkennen. In der Folge wird ein Plugin für den \textit{JavaScript Linter} ESLint erstellt. Diese überprüft die ermittelten Anforderungen für ein erfolgreiches \textit{Tree Shaking} und zeigt gegebenenfalls Warnungen oder Fehler für den Benutzer oder die Benutzerin an. Darauffolgend wird eine statische Code Analyse mit dem \textit{Tree Shaking Plugin} bei sechs NPM Bibliotheken durchgeführt und die Ergebnisse dokumentiert.

Am Ende dieser Arbeit werden anhand der gewonnenen Erkenntnisse Schlussfolgerungen getroffen und ein Ausblick auf mögliche weiterführende Forschung gegeben.